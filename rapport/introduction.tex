L'association Dumbo pour laquelle nous travaillons a pour but de mettre à disposition des associations du monde entier un support numérique solide pour tous leurs besoins tels que le stockage de fichiers, la communication interne, la messagerie mail et autres wiki de référence.

Dumbo a toujours une bonne réputation mais récemment les demandes se multiplient plsu que jamais et il est devenu difficile de répondre à toutes les demandes et l'infrastructure actuelle sature.

Dans ce but, il nous a été demandé de mettre en place une architecture de Cloud pour répondre aux besoins de toutes les associations utilisant les services de Dumbo.

Les technologies et l'architecture existante sont maintenant dépassées et ne répondent plus aux besoins actuels.

Nous avions ainsi pour but de mettre en place une architecture Cloud complète basée sur Docker, l'objectif étant d'avoir un système qui s'étende facilement à tout un datacenter ou même plusieurs distribués sur des sites différents en balance de charge, les containers se plaçant sur les nodes les moins chargés.

Nous avions ainsi l'accès au support de l'école pour pouvoir faire des demandes de créations de machines virtuelles qui représenteraient notre Cloud, avec un total de mémoire vive autorisée de 18Go, un total d'espace de stockage de 220Go et un total de 7vCPU pour faire tourner ces machines. Ayant une SLA de 2h sur la création ou modification de ces machines, il a fallu s'y prendre plsu tôt.

