\subsection{Anticipation}

Durant notre phase d'ingénierie, nous avons pu déduir les principaux problèmes suivants :

\begin{itemize}
\item Evolution trop rapide du système d’information 
\item Des projets d’évolution du système non pas abouti
\item Trop de changement ont été effectués empêchant l’harmonisation des outils et pratiques
\item Insatisfaction client croissante résultant en une perte financière 
\item Organisation non adaptée à l’évolution et le demande des besoins utilisateurs
\item Objectifs d'efficacité, fiabilité, collaboration, résilience et de « time to market » non atteints.
\end{itemize}

Nous pouvons alors formuler des opportunitées que Dumbo peux saisir pour améliorer son infrastructure :

\begin{itemize}
\item Meilleur expérience utilisateur
\item Maintenance simplifiée et plus rapide
\item Montée en charge simplifiée et automatique
\item Datacenters plus écologiques et moins consommateurs d'électricité
\item Administration mutualisée et simplifiée
\item Authentification unifiée des utilisateurs
\end{itemize}

\subsection{Créativité}

\begin{itemize}
\item Nous avons pris d'autres solutions de service que celles proposés de par nos connaissances ou en se renseignant de ce qui se faisait actuellement en entreprises.
\item Nous avons utilisé un reverse proxy avec Traefik nous permettant ainsi de maitriser le port de l'host sur lequel écoutent le conteneur et avoir des informations en temps réel sur ce conteneur.
\item Pour permettre aussi une bonne utilisation de notre solution d'un point de vue utilisateurs nous avons mis en place différents nom de domaine différents.
\end{itemize}

\subsection{Clés du succès}

\begin{itemize}
\item Une méthode de gestion de projet agile afin de se fixer un premier objectif et de s'adapter en fonction de la situation.
\item Une bonne organisation au sein du projet comme une nomenclature précise,une répartition équitable du travail fournit.
\item Une bonne communication au sein du groupe de projet pour voir où en ai le projet et respecter l'architecture du système.
\item Une architecture Open Source qui permet donc une mise a disposition et une modification possible accessible à tous.
\end{itemize}

\clearpage

\subsection{Innovation}

Notre solution est innovante car elle utilise la conteneurisation Docker qui
est une solution innovant en elle-même. Docker permet aujourd'hui de
déployer n'importe quelle application dans n'importe quel environnement.
Elle permet aux applications d'être indépendantes des infrastructures,
et aux développeurs de se libérer des contraintes IT.

Comme en témoigne ce graphique l'innovation des composants Dockers ne
cesse de croitre.

\begin{figue}{1}{ingenierie/pulls-docker.png}
\caption{Évolution des composants téléchargés depuis \url{hub.docker.com}}
\end{figue}

Il faut aussi savoir que les containers Docker, compte tenu de
leur légèreté, sont portables de cloud en cloud.

Nous avons aussi utilisé que des solution Open Source ce qui nous a
permis de pouvoir moduler nos services en fonction de nos besoins donc
ceux du client.

\subsection{Imagination}

Nous avons imaginé une solution innovante et utilisant de nouvelles technologies.
Nous avons tenté de pousser ses technologies tel que Docker, pour en obtenir une architecture innovante et sécurisée.
Durant notre phase de conception nous avons cherché à créer une solution inédite aux problèmes de Dumbo.
Nous avons séparés l'ensemble de nos services en de multiples réseaux permettant de regrouper les services en themes.
Nous avons aussi utilisé Traefik et son système de reverse-proxy dynamique permettant de facilement changer les services disponibles et ainsi d'expérimenter sur de nouveaux services pour toujours plus satisfaire les clients de Dumbo.