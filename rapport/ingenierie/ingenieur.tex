\subsection{Responsabilité Juridique}

Incontestablement, Docker s'adresse à trois grands profils de clients :
les early adopters, les utilisateurs qui découvrent cet environnement et
ceux qui ne l'ont pas encore utilisé. Pour autant, quel que soit son
niveau de maturité, la sécurité reste toujours un obstacle à franchir.

Concrètement, lorsque l'on utilise des « containers », la sécurité
traditionnelle ne suffit pas. On ne sécurise pas une infrastructure «
containérisée » comme on sécurise son infrastructure virtualisée. Il est
donc important de bien comprendre les enjeux de la sécurité dans une
infrastructure « containérisée », ainsi que les risques associés pour
les intégrer efficacement dans sa politique de sécurité.

Evaluation de l'infrastructure containerisée afin de déterminer les
différences de sécurisation de la solution.

\paragraph{Plus de composants}

Dans une infrastructure virtualisée, les flux sont plus au moins
maitrisés. Dans une infrastructure « containérisée », cela se complique
au regard des architectures complexes type « microservices », où le code
est éclaté sur différents composants. Pour les équipes sécurité, il faut
donc s'adapter à cette multitude de flux.

\paragraph{Durée de vie courte des containers}

En comparaison avec les machines virtuelles, un container a une durée de
vie plus courte. Les équipes de développement vont donc constamment
recréer des containers.

\paragraph{Quels changements au niveau des risques ?}

Les risques sont pratiquement identiques à ceux des infrastructures
virtualisées ou classiques. Cependant, certains points sont intéressants
à préciser :

\paragraph{Risques sur les configurations par défaut}

Garder les configurations par défaut amène un risque important dans une
infrastructure conteneurisée. Nous avons donc personnalisé chacune de
nos configuration afin de fournir un service unique et sécurisé
répondant à chacun des besoins de votre entreprise.

\paragraph{Le top ten owasp reste valable}

Avec les containers, on utilise toujours les mêmes protocoles de
communication : les attaques applicatives classiques sont donc
courantes. Nous avons donc créé notre infrastructure en prenant en
compte les dix attaques les plus fréquentes de cette années.

\paragraph{Attaque réseau}

Si un pirate prend la main sur un container, il pourra lancer un scan
réseau et faire des déplacements latéraux. Comme il y a un manque de
visibilité, les équipements de sécurité ne vont pas détecter ce scan, et
le pirate pourra facilement passer d'un container à l'autre.

\subsection{Sécurisation des opérations sur Docker}

Voici les quelques bonnes pratiques que nous avons mis en place :

\begin{itemize}
\item
  Créer une partition physique séparée pour Docker
\item
  Maintenir le système et les images à jour
\item
  Éviter le stockage de secrets au sein d'une image
\item
  Ne pas utiliser n'importe quel registre, nous n'utilisons que le hub
  docker.
\item
  Contrôler les communications entre containers~: nous avons mis en
  place des réseaux spécifiques pour chaque liaison. De cette manière,
  vos services peuvent uniquement communiquer avec leurs propres
  applications.
\item
  Vérifier les sources de construction des images utilisées par les
  conteneurs et assurer un suivi de leurs évolutions.
\end{itemize}

\subsection{Sécurisation des communications}

Nous proposons la mise en place du protocole de communication HTTPS afin
de sécuriser les échanges entre nos serveurs et le client. Grâce à cette
technologie nous pouvons garantir au client que celui-ci communique
uniquement avec nos serveurs lors de l'échange de données
confidentielles. L'HyperText Transfer Protocol Secure~: HTTPS, est la
combinaison du HTTP avec une couche de chiffrement TLS. HTTPS permet au
visiteur de vérifier l'identité du site web auquel il accède, grâce à un
certificat d'authentification émis par une autorité de certification~:
une autorité tierce, réputée fiable et faisant généralement partie de la
liste blanche des navigateurs internet. Il garantit théoriquement la
confidentialité et l'intégrité des données envoyées par l'utilisateur et
reçues du serveur.

\subsection{Sécurisation des données personnelles~: RGPD}

La protection des données personnelles nécessite de prendre des mesures
techniques et organisationnelles appropriées afin de garantir un niveau
de sécurité adapté au risque.

\paragraph{Sensibiliser les utilisateurs}

Faire prendre conscience à chaque utilisateur des enjeux en matière de
sécurité et de vie privée.

\paragraph{Authentifier les utilisateurs}

Reconnaître ses utilisateurs pour pouvoir ensuite leur donner les accès
nécessaires. Par exemple dans le cas de notre solution nous utilisons
LDAP qui est un protocole permettant l'interrogation et la modification
de notre service d'annuaire.

\paragraph{Gérer les habilitations}

Limiter les accès aux seules données dont un utilisateur a besoin.

\paragraph{Tracer les accès et gérer les incidents}

Journaliser les accès et prévoir des procédures pour gérer les incidents
afin de pouvoir réagir en cas de violation de données (atteinte à la
confidentialité, l'intégrité ou la disponibilité).

\paragraph{Protéger le réseau informatique interne}

Autoriser uniquement les fonctions réseau nécessaires aux traitements
mis en place.

\paragraph{Sécuriser les serveurs}

Renforcer les mesures de sécurité appliquées aux serveurs.

\paragraph{Sauvegarder et prévoir la continuité d'activité}

Effectuer des sauvegardes régulières pour limiter l'impact d'une
disparition non désirée de données.

\paragraph{Archiver de manière sécurisée}

Archiver les données qui ne sont plus utilisées au quotidien mais qui
n'ont pas encore atteint leur durée limite de conservation, par exemple
parce qu'elles sont conservées afin d'être utilisées en cas de
contentieux.

\paragraph{Encadrer la maintenance et la destruction des données}

Garantir la sécurité des données à tout moment du cycle de vie des
matériels et des logiciels.

\paragraph{Gérer la sous-traitance}

Encadrer la sécurité des données avec les sous-traitants.

\paragraph{Sécuriser les échanges avec d'autres organismes}

Renforcer la sécurité de toute transmission de données à caractère
personnel.

\paragraph{Protéger les locaux}

Renforcer la sécurité des locaux hébergeant les serveurs informatiques
et les matériels réseaux.

\paragraph{Chiffrer, garantir l'intégrité ou signer}

Assurer l'intégrité, la confidentialité et l'authenticité d'une
information.

\subsection{L'impact sur l'environnement}

Le groupe doit analyser l'impact de la solution sur la société et
l'environnement en s'appuyant par exemple sur une politique Green IT ou
de développement durable

\begin{itemize}
\item
  Les matériels qui ne demandent pas un fonctionnement permanent sont
  systématiquement éteints et non mis en veille. Grâce à la
  containerisation il est possible lors de perte de charge de limiter
  l'utilisation des machines afin de réduire la consommation d'énergie.

  \begin{itemize}
  \item
    Objectifs :

    \begin{itemize}
    \item
      Permettre une réduction de la consommation énergétique supérieure
      à celle relevée lorsque les appareils sont en veille
    \item
      Adopter un comportement responsable
    \end{itemize}
  \end{itemize}
\item
  Les calculs et les tâches sont répartis sur différents serveurs dédiés
  en fonction de la nature de leurs tâches (données, mails, vidéos etc.)

  \begin{itemize}
  \item
    Objectifs :

    \begin{itemize}
    \item
      Maîtriser l'impact sur la consommation énergétique
    \item
      Réduire l'impact écologique des échanges d'information
    \end{itemize}
  \end{itemize}
\item
  Dans le cadre de travaux collectifs, l'usage d'outils collaboratifs en
  ligne est privilégié à l'échange de courriels.

  \begin{itemize}
  \item
    Objectifs :

    \begin{itemize}
    \item
      Réduire l'impact écologique des échanges d'information
    \item
      Sensibiliser les personnels
    \item
      Responsabiliser les personnels
    \item
      Maîtriser la politique GreenIT de l'entreprise
    \end{itemize}
  \end{itemize}
\item
  Anti-SPAM~: Tous les spams sont supprimés

  \begin{itemize}
  \item
    Objectifs :

    \begin{itemize}
    \item
      Limiter l'impact écologique du stockage de l'information
    \item
      Induire des comportements responsables
    \end{itemize}
  \end{itemize}
\end{itemize}
