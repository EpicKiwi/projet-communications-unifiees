\subsection{Sécurisation }

Afin de proposer un environnement de travail sécurisé, nous avons
identifié deux risques potentiels concernant la messagerie et nous les
avons réglés.

Expédier ses emails sans cryptage SSL revient à envoyer un courrier en
langage clair, non chiffré. Ces emails non sécurisés peuvent être
facilement interceptés et lus par un individu non autorisé. À travers
l'utilisation du protocole de transmission TLS, les échanges de
courriers électroniques peuvent être chiffrés. Si l'utilisateur appuie
sur le bouton « envoyer », le message est transmis de manière codée et
sera de nouveau rendu visible chez le destinataire à l'aide de la clé
appropriée.

Le volume considérable d'emails peut affecter les ressources
informatiques. 90 \% des e-mails entrants sont constitués de spams,
supprimer les messages indésirables avant qu'ils n'atteignent le réseau
permet de libérer des ressources informatiques et de réduire les chances
d'être infecté par un cybercriminel. C'est pourquoi nous avons opté pour
une messagerie, sur laquelle nous avons configuré une protection
antispam aux e-mails entrants.

\subsection{Procédures d'exploitation}

\subsubsection{Ajouter un cluster}

Ouvrez un terminal sur la machine sur laquelle vous voulez exécuter
votre nœud.

La commande «~docker swarm join~» va ajouter le nœud sur lequel vous
vous trouvez, dans le cluster spécifié via le token (\emph{Le Token ID
est généré par le nœud manager)} entré en argument.

\begin{lstlisting}
$ docker swarm join <Swarm Token ID>
\end{lstlisting}

\subsubsection{Déployer et/ou mettre à jour un service}

Ouvrez un terminal sur la machine sur laquelle vous voulez exécuter
votre nœud. Puis créer/modifier le fichier «~docker-compose.yml~» dans
lequel figure la configuration du service.

\begin{lstlisting}
$ vim docker-compose.yml
\end{lstlisting}

Une fois le fichier enregistré, utilisez cette commande afin de déployer
le service, en spécifiant le nom du service. Cette commande va
redémarrer seulement les services modifiés/créés

\begin{lstlisting}
$ docker stack deploy -c docker-compose.yml <Service Name>
\end{lstlisting}

Vérifier l'état de vos services~en temps réel.

\begin{lstlisting}
$ watch docker service ls
\end{lstlisting}

\begin{figue}{1}{ingenierie/fig-1.png}
\caption{Kore services}
\end{figue}

\subsubsection{Monter en charge}

Ouvrez un terminal sur la machine sur laquelle votre nœud est exécuté.

La commande scale vous permet de mettre à l'échelle un ou plusieurs
services répliqués, en augmentant ou en diminuant le nombre souhaité de
répliques.

\begin{lstlisting}
$ docker service scale <SERVICE-ID>=<NUMBER-OF-TASKS>
\end{lstlisting}

Vérifiez les informations de votre service.

\begin{lstlisting}
$ docker service ps <SERVICE-ID>
\end{lstlisting}

\subsection{Cycle de vie}

Notre infrastructure possède une certaine tolérance aux pannes, grâce à
Docker Swarm, nous avons définie 3 nœuds manager ce qui permet au
logiciel d'avoir une tolérance aux pannes d'une machine sans impacter
les possibilités d'administration.

Malgré cela un incident peut toujours arriver et c'est pourquoi, la DSI
de l'association Dumbo a à sa disposition tous les fichiers
«~Docker-compose.yml~» comportant la configuration des différents
services de l'environnement numérique. Cela permet par exemple de
revenir à une version antérieure d'un service après une mise à jour
problématique.

De plus le Docker Swarm que nous avons mis en place permet à la DSI de
redéployer ses services et de les partager de façon rapide et efficace
pour pouvoir garantir une reprise d'activité rapide de la production
après un incident non souhaité.

Cette configuration permet également une amélioration continue des
services en permettant une mise à jour simple des services pour tous les
utilisateurs.

Afin de remettre en état ou de mettre à jour un service, référez-vous à
la partie procédures d'exploitation.

\begin{figue}{0.35}{ingenierie/fig-2.png}
\caption{Cycle de vie d'un service}
\end{figue}

\subsection{Amélioration et évolution}

\paragraph{Service de monitoring}

Afin de répondre le plus efficacement possible aux besoins de
l'association et de prévoir d'éventuels problèmes de surcharge, il
serait intéressant d'intégrer une solution de monitoring tel que
Prometheus qui permet de collecter des mesures de plusieurs cibles
préalablement définies.

\paragraph{Système de fichier décentralisé}

Cette évolution permettra de garantir un accès aux documents partagés
sur le réseau pour les utilisateurs, même en cas d'arrêt du serveur
master procédant à la synchronisation des fichiers grâce à une solution
du type Ceph.

\paragraph{Adapter les applications aux besoins des utilisateurs}

Dans une optique d'amélioration continue il serait intéressant de
demander l'avis des utilisateurs concernant les différentes
applications. Ceci permettra de répondre au mieux aux besoins
utilisateurs et d'augmenter la qualité d'expérience.
