\section{Réalisation technique}

Dans le cadre du projet Dumbo, nous devions réfléchir à une infrastructure permettant de déployer un grand nombre de services pour de nombreux clients.
Cela pose de nombreux problèmes comme la gestion des différents espaces d'entreprise, la gestion de la charge au sein d'un cluster de machines et de multiple clients.

\subsection{Infrastructure orientée containers}

Nous avons déployé nos services au sein de containers Docker.
Docker est une téchnologie permettant d'enfermer les applications déployées dans un système de fichiers isolé du reste du système.
Cela permet aux différentes applications de s'éxécuter dans un environnement sécurisé indépendemment des autres applications qui pourraientt influencer le reste de l'installation.
Cela permet aussi de séparer les différents services et de les redémarrer facilement en cas d'erreur.
En effet, lors du crash d'un container, il suffit de le redémarrer ce qui à pour effet de réinitialiser son contenu et ainsi de reprendre sur une base saine.
La persistance des données s'effectue par le montage de dossiers de stockage depuis la machine hôte.
Ainsi, les données utilisateurs sont constantes et les services peuvent redémarrer à tout momment.

\subsection{Orchestration multi-nodes}

\begin{figue}{1}{realisation/archi.pdf}
\caption{Schéma de l'intrastructure d'orchestration}
\end{figue}

Notre infrastructure comprend plusieurs machines permettant de diffuser la charge sur l'ensemble d'un cluster de machines.
On appelle l'ensemble de ces machines communiquantes, un Cluster et chaque membre de ce Cluster est appelé Noeud.

Le stockage est assuré par un disque présent sur le Noeud principal (Noeud Manager) disposant d'un grand espace de stockage.
Ce disque est alors exporté via le protocole NFS vers les autres noeuds pour conserver un stockage synchronisé sur l'ensemble du cluster.
Ainsi, les containers disposent d'un accès au stockage peu importe de leur emplacement dans le Cluster.

Un Cluster implique un système d'orchestration.
Un tel système gère l'ensemble des containers exécutés sur chaque Noeud et définis quel container doit être démarré ou arrêté et sur quel machine.
Dans notre cas, nous avons choisi Docker Swarm pour assurer cette Orchestration.
Apres avoir constitué le cluster, il suffit de définir de de configurer les services qui seront déployés au sein de celui-ci.

Un service est une abstraction d'un container.
Un service définis le squelette d'un container mais aussi le nombre de container qu'il faut déployer simultanément et d'autres paramètre permettant, notemment, de configurer le Reverse Proxy.
Une fois les services définis, Docker Swarm se charge de lui même de démarrer les containers sur les différents Noeuds et de les connecter ensemble.

\subsection{Services proposés}

\paragraph{Pad} Un Pad est un fichier texte très simple permettant l'utilisation d'une mise en forme basique tel que la mise en gras, la mise en italique, certains titres, etc.
Un Pad peut aussi être partagé pour une édition en collaboration, ainsi plusiuers perosnnes de l'entreprise peuvent le lire et l'éditer en simultanée.
Dans le cadre d'une entrperise, cela permet de partager rapidement des informations sans envoyer d'email ou de fichier Word.
Cette techologie est très utile dans le cadre d'une rédaction de document en équipe, cela permet aux divers memebres de l'équipe de débattre et de modifier le texte sans se soucier de la mise en forme mais de se concentrer sur le fond.
Pour la mise en place de ce service sur \url{pad.kore.sh}, nous avons choisi la technologie \emph{Etherpad} qui est la plus avancée actuellement et qui propose de nombreuses fonctionnalités.

\paragraph{Gestion des fichiers} La gestion des fichiers permet le stockage et le partage de fichiers dans un espace personnel pour chaque utilisateur.
Chaque utilisateur dispose de ses propres fichiers qu'il peut partager, diffuser et modifier en ligne.
Pour mettre a disposition des utilisateurs ce service sur \url{files.kore.sh}, nous avons déployé un serveur \emph{NextCloud} permettant de partager et de gèrer des fichiers au travers du protocole WebDAV.
En plus de ce serveur de fichiers, nous avons mis en place une solution \emph{OnlyOffice} permettant d'éditer en ligne et à plusieurs les documents de la suite Office Microsoft.
cela comprend un tableur, un traitement de texte mais aussi un créateur de présentations.
On peut aussi noter que les fichiers NextCloud sont accessibles depuis un téléphone au travers de l'application NextCloud permettant ainsi un accès aux données même en mobilité.

\paragraph{Messagerie} La messagerie est un élément important d'une entrperise, elle permet la communication avec l'ensemble des collaborateurs mais aussi avec les clients et toute autre personne.
Nous avons mis en place deux éléments pour la gestion de cette messagerie.
Dans un premier temps, le serveur des messagerie \emph{Postfix} et \emph{Dovecot} permettent d'envoyer et de recevoir les messages au travers de protocols SMTP et IMAP.
Dans un deuxième temps, nous avons déployé une application de messagerie Web \emph{SOGo} accessible à \url{mail.kore.sh/SOGo} gèrant à la fois les messages par la connexion au services précédemment mis en place mais aussi un Agenda et un carnet d'adresses.

\paragraph{Wiki} Un Wiki est un logiciel permettant à chacun de créer et modifier une base de connaissance.
Ce wiki est ouvert à tout, même aux modifications anonymes et permet de constituer une connaissance de groupe accessible en tout temps.
Un système wiki est très utile pour gèrer les problèmes récurrents d'une entreprise tant au niveau des système d'information que d'informations générales sur le fonctionnement de l'entreprise.
Nous avons utilisé le système \emph{Gitit} sur \url{wiki.kore.sh} pour ce service.
Gitit dispose d'une intrface web de modification et gère son historique de modification dans un repository git, cela rend son utilisation très simple et permet aussi une modification hors ligne en clonant ce repository sur sa propre machine.

\paragraph{Chat} Le chat d'entreprise permet de communiquer efficacement entre membres d'une équipe, d'un service et même d'une entreprise.
L'application Rocket.chat permet de disposer de canaux de communication entre tout les membres d'une équipe permettant une communication fluide et bien plus rapide que l'adresse Email.
Ces systèmes de communication se sont généralisés dans les entreprise et permettent bien plus de fonctionnalités qu'un simple échange d'email.
Pour mettre en place cette application sur \url{chat.kore.sh}, nous avons utilisé le système Rocket.chat.
Ce système est plus avancé des systèmes de chat d'entreprise open source et est aussi disponible sur mobile.

\paragraph{Base de données} La base de données de l'entreprise permet un stockage de données arbitraires et leur recherche pour une utilisation ultérieur.
Pour cette utilisation, nous avons choisi le logiciel \emph{CouchDB} permettant un stockage de données sans schéma défini.
CouchDB dispose d'une API REST permettant de communiquer facilement avec un client arbitraire comme un script d'entreprise, des capteurs IoT, etc.
CouchDB dispose aussi d'une interface utilisateur sur \url{db.kore.sh/_utils} permettant une consultation des données par des utilisateur humains.

\paragraph{Annuaire} Un service d'annuaire permet de centraliser la gestion des utilisateurs de l'entreprise.
Nous avons utilisé cette technologie pour garantir qu'un collaborateur créé dans la base de données de l'annuaire dispose d'un compte sur le Gestionnaire de fichiers, le Chat, la Messagerie, le réseau social et le forum.
La base de données gère les noms d'utilisateurs et mots de masse de manière centralisée permettant un modification bien plus aisée des comptes utilisateurs mais aussi une sécurité plus importante car les mots de passes des utilisateur ne sont pas présents en plusieurs endroits, potentiellement vulnérables.
Pour la mise en place de cet annuaire, nous avons utilisé le logiciel OpenLDAP qui est une référence dans le domaine.

\paragraph{Audioconférence et VoIP} Les systèmes de VoIP permettant aux collaborateurs de communiquer par voie vocal ou vidéo entre eux.
Cela passe par les protocol SIP et RTC que nous avons déployés avec l'application open source \emph{Asterisk}.
Chaque collaborateur peut alors utiliser une application de VoIP (appelée SoftPhone) sur smartphone ou sur PC.
Il est aussi possible de configurer des téléphones physiques pour utiliser le système de VoIP mis en place.

\paragraph{Réseau social} Un réseau social d'entreprise permet une communication interne de moindre importance mais permattant aux collaborateurs de partager sur des sujets divers de manière convival et informel.
Nous avons inclus le réseau social d'entreprise dans nos services sur \url{social.kore.sh}.

\paragraph{Forum} Un forum est un site web permettant aux collaborateur de discuter autours de sujets définis.
Ce service est très utile en entreprise pour encourager l'entraide entre collaborateurs.
Cela permet aussi de faciliter les dialogues avec le support technique sur les difficultés que peuvent rencontrer les utilisateurs lors de leur utilisation du système d'information.
Nous avons déployé un forum sur \url{forum.kore.sh}.

\paragraph{Exposition des services} La majorité de ces services sont accessibles sur le Web.
Le port définis pour la communication HTTP étant 80 et 443 pour sa version sécurisée, il est impossible d'exposer plusieurs applications sur la même addresses IP.
Nous avosn donc mis en place un Reverse proxy se positionnant en entrée de toutes les communications vers les services web et distribuant les requètes aux services concernés sur la base du nom de domaine utilisé pour y accéder.
Nous avons utilisé le reverse proxy \emph{Traefik} permettant une configuration dynamique par l'observation du cluster.
En effet, apres avoir déployé un service web, il suffit d'ajouter 4 labels à ce service pour configurer son acces depuis l'extérieur au travers de traefik.
Cela rend le déploiement de services très simple et dynamique.
Enfin, Trafik dispose d'un système de load balancing permettant de diffuser la charge des requètes sur plusieurs containers d'un même service.

\subsection{Structure des services}

Au sein de notre cluster, nous avons déployé ces services et leur dépendances avec une forte séparation garantissant une disponibilité même en cas de panne d'un service.
L'infrastructure est aussi divisé en de nombreux réseaux permettant de séparer les usages de de chaque service.
<<<<<<< HEAD
=======

\begin{figue}{0.94}{realisation/services.pdf}
\end{figue}

\clearpage
>>>>>>> 20e847a823f7d76f070f08d765e70f904783c1b5
